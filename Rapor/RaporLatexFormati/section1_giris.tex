\section{GİRİŞ}

Yapay Zeka, endüstrileri, teknolojiyle etkileşimleri ve dolaylı olarak insan hayatını yeniden şekillendiren bir teknolojidir. Temelde, sistemlerin verilerden öğrenmesini sağlayan makine öğrenimi yatmaktadır. Makine öğreniminin yapay zekadaki önemi, büyük verilerden görüş elde etme, veriye dayalı karar alma ve bu kararlar ile çeşitli avantajlar elde etme yeteneğinden kaynaklanmaktadır.

Makine öğrenimi, görevleri otomatikleştirerek, iş akışlarını düzene sokarak ve insan müdahalesini en aza indirerek endüstriler genelinde otomasyonu ve verimliliği arttırmaktadır. Örneğin bir üretim hattında önceki ürünlerden edindiği bilgiler ile yeni ürünün kalite kontrol süreçlerinde insan müdahalesini ortadan kaldırırken; bir diğer yandan sağlık sektöründe yıllardır biriken hastalık tanı verileri ile yeni muayene süreçlerinde hızlı ve doğru hastalık tanımlaması yaparak, kişiye özel tedaviler önerebilir.

Bunların yanında makine öğrenimi, doğal dil işleme, görüntü tanımlama - işleme, ses tanımlama - işleme gibi alanlarda akıllı sistemlere güç sağlar. Böylece sanal asistanların, sohbet robotlarının, öneri sistemlerinin vs. dil çevirisinden yüz tanımaya kadar kullanıcı deneyimlerini geliştirmesine ve insan-bilgisayar etkileşimini kolaylaştırmasına yardımcı olur.

Makine öğrenimi sistemlerinin yinelemeli bir doğası vardır. Bu durum yapay zeka sistemlerinde sürekli iyileştirmeye ve adaptasyona imkan sağlar. Modeller zaman içinde gelişerek tahminlerini ve performanslarını iyileştirebilir. Böylece her bir aşamada daha karmaşık, daha kaotik bir sorunun veya görevin üstesinden gelebilecek bir kapasiteye ulaşır.

Özetle, makine öğrenimi yapay zekanın temel taşıdır. Makine öğrenimi veri analizini, süreçlerin otomasyonunu, insan - bilgisayar etkileşimini ve sürekli öğrenmeyi mümkün kılar. Etkisi dolaylı veya direk bir şekilde doğayı, insanları, endüstrileri kapsamaktadır. Gelecekteki inovasyon süreçleri için neredeyse limitsiz bir potansiyele sahiptir.