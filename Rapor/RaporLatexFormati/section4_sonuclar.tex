\newpage
\section{SONUÇLAR VE GELECEK ÇALIŞMALAR}

Bu raporda sınıflandırma işlemleri için \href{https://www.kaggle.com/datasets/iammustafatz/diabetes-prediction-dataset/data}{Diabetes Prediction Dataset}, regresyon işlemleri için \href{https://www.kaggle.com/datasets/anmolkumar/house-price-prediction-challenge/data}{House Price Prediction Challenge}, kümeleme işlemleri için \href{https://www.kaggle.com/datasets/datascientistanna/customers-dataset/data}{Shop Customer Data} veri setleri kullanılmıştır.

Sonuçlarda şu maddeler görülmüştür:
\begin{itemize}
\item Sınıflandırma işlemlerinde bireylerin diyabet olma durumu 90\%'nin üzerinde bir doğruluk ile tahmin edilmektedir.
\item Regresyon işlemlerinde veri setindeki ortalama ev fiyatı 65 - 70'dir. Analizler sonucu Mean Absolute Error 10 mertebesindedir. Yani bir evin fiyatı ortalama 15\% hata ile tahmin edilebilmektedir.
\item Kümeleme işlemlerinde verinin PCA'ye sokulmadan önceki hali ile "detransform" edilen veri karşılaştırıldığında, verinin kümeleme işleminde doğru temsil edilme oranının yüksek olduğu rahatlıkla görülebilmektedir.
\end{itemize}

Tüm bunlar göz önüne alındığında, makine öğrenmesinin hayatın çok farklı alanlarında, birbirlerinden farklı amaçlar doğrultusunda, yüksek doğrulukla hizmet edebildiği söylenebilir. Bu güçlü teknolojinin kullanımında en temel limit insanın hayal gücü olabilir.

Gelecek çalışmalarda:
\begin{itemize}
\item Sınıflandırma ve regresyon konusunda daha karmaşık veri setleri kullanılabilir.
\item Kümeleme işlemlerinde hedef sütun olan veri setleri denenebilir.
\item Bu program için hazırlanan arayüz çok daha profesyonel bir hale getirilebilir.
\end{itemize}